\documentclass[11pt, a4paper]{article}

% General document formatting
\usepackage[margin=0.7in]{geometry}
\usepackage[parfill]{parskip}
\usepackage{url, hyperref}

% Related to math
\usepackage{amsmath,amssymb,amsfonts,amsthm}

% encoding and language
\usepackage{lmodern}
\usepackage[slovene]{babel}
\usepackage[utf8]{inputenc}
\usepackage[T1]{fontenc}

% multiline comments
\usepackage{verbatim}

% dashed underline
\usepackage{ulem}

% images
\usepackage{graphicx}
\graphicspath{ {./images/} }

% theorems
\theoremstyle{definition}
\newtheorem{counter}{Counter}[section] % not for use
\newtheorem{defn}[counter]{Definicija}
\newtheorem{lemma}[counter]{Lema}
\newtheorem{conseq}[counter]{Posledica}
\newtheorem{claim}[counter]{Trditev}
\newtheorem{theorem}[counter]{Izrek}
%%
\theoremstyle{remark}
\newtheorem*{ex}{Primer}
\newtheorem*{rem}{Opomba}

% I like my squares DARK
\renewcommand\qedsymbol{$\blacksquare$}

% common number sets for convenience purposes
\newcommand{\N}{\mathbb{N}}
\newcommand{\Z}{\mathbb{Z}}
\newcommand{\Q}{\mathbb{Q}}
\newcommand{\R}{\mathbb{R}}

% fix bordermatrix
\usepackage{etoolbox}
\let\bbordermatrix\bordermatrix
\patchcmd{\bbordermatrix}{8.75}{4.75}{}{}
\patchcmd{\bbordermatrix}{\left(}{\left[}{}{}
\patchcmd{\bbordermatrix}{\right)}{\right]}{}{}

\begin{document}	
	\title{Algebrske strukture - zapiski predavanj prof. Klav"zarja}
	\author{Yon Ploj}
	\date{2. semester 2021}
	\maketitle
	
	\tableofcontents
	\vspace{2cm}

	% 13. 04. 2021
	\subsection{Lastnosti operacij}
	\begin{defn}[Asociativnost]
		\[ (a \cdot b) \cdot c = a \cdot (b \cdot c) \]
	\end{defn}
	\begin{defn}[Komutativnost]
		\[ a \cdot b = b \cdot a \]
	\end{defn}
	\begin{defn}[Enota]
		\[ a \cdot e = e \cdot a = a \]
	\end{defn}
	\begin{theorem}
		Enota je enoli"cna.
	\end{theorem}
	\begin{proof}
		Predpostavimo, da obstajata dve enoti $e_1$ in $e_2$.
		Ker je $e_1$ enota, je $e_1 \cdot e_2 = e_2$.
		Ker je $e_2$ enota, je $e_1 \cdot e_2 = e_1$.
		Sledi, da je $e_1 = e_2$.
	\end{proof}

	\begin{defn}[Inverz / Obratna vrednost $a$]
		\[ a \cdot a^{-1} = a^{-1} \cdot a = e \]
	\end{defn}
	\begin{rem}
		Inverz abstraktnega mno"zenja ozna"cujemo z $a^{-1}$, inverz abstraktnega se"stevanja pa z $-a$.
	\end{rem}
	\begin{theorem}
		Inverz je enoli"cen.
	\end{theorem}
	\begin{proof}
		Predpostavimo, da obstajata dva inverza $b_1$ in $b_2$.
		\[ b_1 = b_1 \cdot e = b_1 \cdot (a \cdot b_2) = (b_1 \cdot a) \cdot b_2 = e \cdot b_2 = b_2 \]
	\end{proof}

	\section{Algebrske strukture}
	\begin{defn}[Notranja operacija mno"zice $A$]
		\[ f: A \times A \rightarrow A\]
		Z infiksno notacijo ozna"cujemo $f(a,b)$ kot $a \cdot b$ ali $ab$
	\end{defn}
	\begin{defn}[Algebrska struktura]
		Mno"zica z vsaj eno notrajno operacijo
	\end{defn}
	\begin{defn}[Grupoid]
		Mno"zica z notrajno operacijo. $(M, \cdot)$
	\end{defn}
	\begin{defn}[Polgrupa]
		Asociativen grupoid.
	\end{defn}
	\begin{defn}[Monoid]
		Polgrupa z enoto.
	\end{defn}
	\begin{defn}[Grupa]
		Monoid, kjer je vsak element obrnljiv.
	\end{defn}
	\begin{defn}[Abelova grupa]
		Komutativna grupa.
	\end{defn}

	\begin{comment}
	\begin{defn}[Kolobar]
		Mno"zica z 2 operacijama $(M, +, \cdot)$ \\
		kjer je $(M, +)$ abelova grupa in $(M, \cdot)$ monoid.
		% kokoid
	\end{defn}
	\begin{defn}[Obseg]
		Kolobar, kjer so neni"celni elementi grupa za $\cdot$
	\end{defn}
	\begin{defn}[Polje]
		Komutativni obseg
	\end{defn}
	\begin{defn}[Modul]
		Kolobar z abelovo grupo $((M, +, \cdot), (V, \oplus))$
	\end{defn}
	\begin{defn}[Vektorski prostor]
		Modul, kjer je $(M, +, \cdot)$ polje.
	\end{defn}
	\end{comment}
	
	\subsection{Mno"zica $\Z_n$}
	\begin{defn}[Kongruenca]
			$a$ in $b$ sta kongruentna po modulu $m$ ntk. obstajajo $p,q,r \in \Z_n$, da velja:
			\[ a = p*m + r \]
			\[ b = q*m + r \]
			\[ r < p \quad \land \quad r < q \]
	\end{defn}
	Relacija kongruence je ekvivalen"cna, zato razdeli $\Z_n$ na ekvivalen"cne razrede ostankov: $\lbrace 0, 1, \ldots, n-1 \rbrace$
	
	\begin{rem}
		V nadaljevanju bomo uporabljali operaciji $+_{n}$ in $\cdot_{n}$ kot se"stevanje/mno"zenje po modulu $n$.
	\end{rem}
	
	\begin{claim}
		$(\Z_n, +_n)$ je grupa
	\end{claim}
	\begin{claim}
		$(\Z_n, \cdot_n)$ je monoid
	\end{claim}
	$x \in \Z_n$ je obrnljiv $\iff$ $x \perp m$. Zato velja, da so vsi elementi v $\Z_p$ (kjer je $p$ pra"stevilo) obrnljivi. $\Z_p$ je torej grupa.
	
	% 20. 4. 2021 (blaze it)
	\section{Grupe}
	\begin{defn}[Cayleyeva tabela]
		Tabela, ki prikazuje definicijo operacije v kon"cnem monoidu.
		$$
		\bbordermatrix{
			\cdot & i & r & s & x & y & z \cr
				i & i & r & s & x & y & z \cr
				r & r & s & i & y & z & x \cr
				s & s & i & r & z & x & y \cr
				x & x & z & y & i & s & r \cr
				y & y & x & z & r & i & s \cr
				z & z & y & x & s & r & i \cr
		}
		$$
	\end{defn}
	\begin{rem}
		V Cayleyevi tabeli grupe so vsi elementi v vsakem stolpcu in vsaki vrstici med seboj razli"cni (Cayleyeva tabela je latinski kvadrat reda $n$). To sledi iz izreka \ref{praviloKrajsanja}
	\end{rem}
	\begin{theorem}[Pravilo kraj"sanja]\label{praviloKrajsanja}
		"Ce je $(G, \cdot)$ grupa in $a, b, c \in G$, potem velja:
		\[ ba = ca \implies b = c \]
		\[ ab = ac \implies b = c \]
	\end{theorem}
	\begin{proof}
		Naj bo $ba = ca$. Na desni pomno"zimo z $a^{-1}$ in zaradi asociativnosti dobimo:
		\[ (ba)a^{-1} = (ca)a^{-1} \]
		\[ b(aa^{-1}) = c(aa^{-1}) \]
		\[ be = ce \]
		\[ b = c \]
	\end{proof}

	\begin{defn}[Red elementa]
		Naj bo $(G, \cdot)$ kon"cna grupa. Tedaj je red elementa $a \in G$ najmanj"se naravno "stevilo $n$, za katerega velja
		\[ a^n = e \]
		"Ce je $G$ neskon"cna in za $a$ ne obstaja noben $n$ da velja $a^n = e$, je red $a$ neskon"cno.
	\end{defn}
	\begin{claim}
		Red elementa je dobro definiran
	\end{claim}
	\begin{proof}
		Poglejmo zaporedje: $a^1, a^2, \cdots, a^{k+1}$, kjer je $k=|G|$. Zaporedje ima $k+1$ elementov, na"sa grupa pa jih ima $k$.
		Po dirichletovem na"celu % osnovni princip kombinatorike
		\[\exists p,q: (p \neq q \land (\text{B"SS } p < q) \land a^p = a^q) \]
		Tedaj
		\[ e = (a^p)(a^p)^{-1} = (a^q)(a^p)^{-1} = a^q a^{-p} = a^{q-p} \]
		Sledi $a^{q-p} = e$, kar smo "zeleli pokazati.
	\end{proof}
	\begin{rem}
		Red enote je $1$ in ker je enota enoli"cna, je enota edini element reda $1$.
	\end{rem}
	
	\section{Podgrupe}
	\begin{defn}[Podgrupa]
		Naj bo $(G, \cdot)$ grupa. Tedaj je $H \subseteq G$ podgrupa, "ce je $(H, \cdot)$ tudi grupa. Pri tem je operacija obakrat ista. Ozna"cimo $H \leq G$.
	\end{defn}
	\begin{defn}[Prava podgrupa]
		Naj bo $(H, \cdot)$ podrgupa $(G, \cdot)$. "Ce je $H \subset G \text{ (torej } H \neq G$), je $H$ prava podgrupa $G$. Ozna"cimo $H < G$.
	\end{defn}
	
	\begin{ex}[Trivialna podgrupa]
		Za vsako grupo $G$ velja $G \leq G$ in $\lbrace e \rbrace \leq G$.
	\end{ex}
	
	\begin{ex}
		$(\Q^+, \cdot) < (\R^+, \cdot)$
	\end{ex}
	\begin{ex}
		$F := \lbrace f: \R \rightarrow \R \rbrace$. $(F, +)$ je grupa. \\
		$C := \lbrace f: \R \rightarrow \R ; f \text{ je zvezna}\rbrace$. $(C, +)$ je grupa. \\
		$(C, +) < (F, +)$
	\end{ex}

	\begin{theorem}[Glavni izrek o podgrupah]\label{glavniIzrekPodrgup}
		Naj bo $(G, \cdot)$ grupa in $\emptyset \neq H \subseteq G$. Tedaj je $(H, \cdot)$ podgrupa v $(G, \cdot)$ natanko tedaj, ko
		\[ \forall x,y \in H: (x^{-1}y \in H) \]
	\end{theorem}
	\begin{proof}
		($\Rightarrow$) Naj bosta $x,y \in H$. Ker je $(H, \cdot)$ podgrupa in s tem sama zase grupa, je tudi $x^{-1} \in H$. Zato je tudi $x^{-1}y \in H$.
		\\
		($\Leftarrow$) Naj $\forall x,y \in H: (x^{-1}y \in H)$.
		\begin{itemize}
			\item asociativnost \\
			"ce so $x,y,z \in H$, potem so tudi $x,y,z \in G$. Ker v $G$ velja asociativnost, velja tudi v $H$.
			
			\item enota \\
			Ker je $H \neq \emptyset$, $\exists x \in H$. Postavimo $y = x$. Potem je tudi $x^{-1}x = e \in H$.
			
			\item inverz \\
			Vemo, da je $e \in H$. Naj bo $x \in H$. Postavimo $y = e$: $x^{-1}y \in H \implies x^{-1}e \in H \implies x^{-1} \in H$.
			
			\item zaprtost \\
			$x, y \in H$. Vemo "ze, da je $x^{-1} \in H$, zato je tudi $(x^{-1})^{-1} \in H$. Zato je $xy = (x^{-1})^{-1}y \in H$.
		\end{itemize}
	\end{proof}
	
	Za kon"cne grupe je kriterij "se enostavnej"si:
	\begin{theorem}
		Naj bo $(G, \cdot)$ kon"cna grupa in $\emptyset \neq H \subseteq G$. Tedaj je $(H, \cdot) \leq (G, \cdot) \iff (x,y \in H \implies xy \in H)$
	\end{theorem}
	\begin{proof}
		Dokaz je tako zelo enostaven, da ga ne bomo "sli dokazovat. Glavna ideja je, da malo gledate ta zaporedja in potem dobite neke zaklju"cke. %ok boomer
	\end{proof}
	
	\begin{defn}[Cikli"cna podgrupa]
		Naj bo $(G. \cdot)$ grupa in $a \in G$. Potem naj bo
		\[ \langle a \rangle := \lbrace a^n: n \in \Z \rbrace \]
		Podgrupa $(\langle a \rangle, \cdot)$ je cikli"cna podgrupa v $G$, generirana z enoto $a$.
	\end{defn}
	\begin{claim}
		"Ce je $(G, \cdot)$ grupa in $a \in G$, potem je
		\[ (\langle a \rangle, \cdot) \leq (G, \cdot) \]
	\end{claim}
	\begin{proof}
		Ker je $a^1 = a$, je $a \in \langle a \rangle$, torej $\langle a \rangle \neq \emptyset$. Naj bosta sedaj $a^n, a^m \in \langle a \rangle$.
		Ker je \[(a^n)^{-1}a^m = (a^{-1})^na^m = a^{m-n} \in \langle a \rangle \]
		je po glavnem izreku potem $(\langle a \rangle, \cdot)$ podgrupa grupe $G$.
	\end{proof}

	\begin{ex}
		$(\Z_{12}, +_{12})$
		\\
		$\langle 3 \rangle = \lbrace 3, 6, 9, 0 \rbrace$
		\\
		$(\lbrace 0, 3, 6, 9 \rbrace, +_{12}) \leq (\Z_{12}), +_{12})$
	\end{ex}

	\begin{defn}[Center grupe]
		Naj bo $(G, \cdot)$ grupa. Potem je $Z(G)$ center grupe $G$ podmno"zica z elementi, ki komutirajo z vsemi elementi v $G$.
		\[ Z(G) = \lbrace a \in G: \forall x \in G(ax=xa) \rbrace \]
	\end{defn}

	\begin{rem}
		"Ce je $G$ abelova, je $Z(G) = G$.
	\end{rem}
	\begin{theorem}
		"Ce je $(G, \cdot)$ grupa, potem je $(Z(G), \cdot) \leq (G, \cdot)$.
	\end{theorem}
	\begin{proof}		
		Poka"zimo najprej, da $a \in Z(G) \implies a^{-1} \in Z(G)$. "Ce $a$ komutira z vsemi $x \in G$, potem tudi $a^{-1}$ komutira z vsemi $x \in G$:
		\[ a^{-1} \cdot / \quad ax = xa \quad / \cdot a^{-1} \]
		\[ a^{-1}axa^{-1} = a^{-1}xaa^{-1} \]
		\[ (a^{-1}a)xa^{-1} = a^{-1}ax(a^{-1}) \]
		\[ xa^{-1} = a^{-1}x \]
		
		Sedaj pa "se $a^{-1}b \in Z(G)$:
		\[ (a^{-1}b)x = a^{-1}(bx) = a^{-1}(xb) = (a^{-1}x)b = (xa^{-1})b = x(a^{-1}b) \]
		Po izreku \ref{glavniIzrekPodrgup} je to zadosti.
	\end{proof}

	% 4. 5. 2021
	\section{Cikli"cne in permutacijske grupe, izomorfizmi}
	\begin{defn}[Cikli"cna grupa]
		Naj bo $(G. \cdot)$ grupa in $a \in G$. "Ce velja 
		\[ \langle a \rangle = G \]
		potem je $G$ cikli"cna grupa, $a$ pa njen generator.
	\end{defn}
	
	\begin{ex}
		$(\Z, +)$ je cikli"cna grupa z generatorjema $1$ in $-1$.
	\end{ex}

	\begin{ex}
		$(\Z_9, +)$ je cikli"cna grupa. $1$ je gotovo generator, obstajajo pa tudi drugi (recimo $4$). Na"steli jih bomo kasneje.
	\end{ex}

	\begin{theorem}
		Naj bo $G$ grupa in $a \in G$.
		
		\begin{enumerate}
			\item "Ce ima $a$ neskon"cen red, potem so vse potence $a^n$ med seboj paroma razli"cne.
			\item "Ce ima $a$ kon"cen red, potem je
			\[ \langle a \rangle = \lbrace e, a, a^2, \ldots, a^{n-1} \rbrace \]
			Nadalje, $a^i = a^j$ velja natanko tedaj, ko $n | (i-j)$.
		\end{enumerate}
	\end{theorem}
	\begin{proof}$ $
		\begin{enumerate}
			\item Naj ima $a$ neskon"cen red. Opazujmo $a^i$ in $a^j$, $i \neq j$. "Ce bi veljalo $a^i = a^j$, bi $a^{i-j} = e$. Ampak $i \neq j$: to bi pomenilo, da ima $a$ kon"cen red.
		
			\item Naj ima $a$ kon"cen red $n$.
			\[ X := \lbrace e, a, a^2, \ldots, a^{n-1} \rbrace \]
			Poka"zimo $\langle a \rangle = X$.
			O"citno je $X \subseteq \langle a \rangle $, saj $a^i \in X \overset{\text{def.}}{\implies} a^i \in \langle a \rangle $.
			Poka"zimo torej, da $ \langle a \rangle \subseteq X$, oziroma:
			\[ a^k, k \in \Z \implies a^k \in X \]
			Po izreku o deljenju:
			\[ k = p \cdot n+r \quad 0 \leq r < n \]
			\[ a^k = a^{p \cdot n+r} = a^{pn} \cdot a^r = (a^n)^p \cdot a^r = e^p \cdot a^r = a^r \]
			ampak $ 0 \leq r < n $, torej $a^k = a^r \in X $
			
			\item $a^i = a^j \iff n | (i-j)$:
			\[ i-j = p \cdot n + r \]
			($\Rightarrow$) Naj bo $a^i = a^j$. Tedaj
			\[ e = a^{i-j} = a^{p \cdot n + r} = a^p \cdot a^r = a^r \quad r < n \]
			Ker je red $a$ enak $n$ in je $r < n$, velja $r = 0$. Torej $i-j = p \cdot n$, oziroma $n | (i-j)$. \\
			
			($\Leftarrow$) Naj $n | (i-j)$.
			\[ i-j = p \cdot n + r \quad (0 \leq r < n) \overset{n | (i-j)}{\implies} r = 0 \implies i-j=p \cdot n\]
			\[ a^i = a^{p \cdot n + j} = (a^n)^p \cdot a^j = a^j \]
		\end{enumerate}
	\end{proof}

	\begin{conseq}
		Naj bo $G$ grupa in $a \in G$ reda $n$. "Ce $a^k = e$, potem $n | k$.
	\end{conseq}
	\begin{proof}
		\[ a^0 = e = a^k \]
		Poprej vemo, da $a^i = a^j \iff n | (i-j)$. Vstavimo $i=k$, $j=0$, dobimo $n | (k-0)$, torej $n | k$.
	\end{proof}

	\begin{theorem}
		Naj bo $G$ cikli"cna grupa in $a \in G$ element reda $n$. Potem je $G = \langle a^k \rangle$ natanko tedaj, ko je $(n,k) = 1$
	\end{theorem}
	\begin{ex}
		\[ (\Z_9, +) = \langle 1 \rangle = \langle 9 \rangle \]
		\[ \Z_9 = \langle 1^k \rangle \iff \langle k,9 \rangle = 1 \]
		Torej generatorji so $1, 2, 4, 5, 7, 8$.
	\end{ex}

	\subsection{Permutacijske grupe}
	\begin{defn}[Permutacija mno"zice $A$]
		Je bijekcija $A \rightarrow A$.
	\end{defn}
	%$\Pi$ torej interpretiramo kot $\Pi: [n] \rightarrow [n]$
	
	\begin{defn}[Permutacijska grupa]
		Je mno"zica permutacij, ki za komponiranje preslikav tvorijo grupo.
	\end{defn}
	\begin{defn}[Simetri"cna grupa $S_n$]
		"Ce vzamemo vse permutacije mno"zice $[n]$, dobimo simetri"cno grupo $S_n$. Ta grupa ni abelova.
	\end{defn}
	\begin{claim}
		$|S_n| = n!$
	\end{claim}
		
	\begin{claim}
		Vsako permutacijo lahko enoli"cno (do vrstnega reda faktorjev natan"cno) zapi"semo kot produkt disjunktnih ciklov.
		% proof: go fuck yourself
	\end{claim}
	\begin{proof}
		Lmao you thought
	\end{proof}
	\begin{claim}
		Vsako permutacijo lahko zapi"semo kot produkt transpozicij.
		% proof: go fuck yourself
	\end{claim}

	\begin{claim}
		Neko permutcijo lahko zapi"semo bodisi samo kot produkt sodo ali liho "stevilo transpozicij. Pravimo, da je permutacija liha ali soda.
	\end{claim}

	\begin{defn}[Alternirajo"ca grupa $A_n$]
		Je grupa vseh sodih permutacij mno"zice $[n]$.
	\end{defn}
	Dokaz da je to grupa lahko naredite sami.
	
	\begin{theorem}
		"Ce je $n > 1$, potem je $|A_n| = \frac{n!}{2}$
	\end{theorem}
	\begin{proof}
		Vzemimo poljubno liho permutacijo $\Pi$.
		\[ \underset{\text{liha}}{\Pi} \quad \underset{\text{injektivno}}{\rightarrow} \quad \underset{\text{soda}}{(12)\cdot\Pi} \]
		\[ \forall \Pi, \Sigma \text{ lihi: } \Pi \neq \Sigma \implies (12)\cdot\Pi \neq (12)\cdot\Sigma \]
		"Stevilo sodih permutacij $\geq$ "stevilo lihih permutacij. Z obratnim razmislekom ugotovimo, da je "stevilo sodih $=$ "stevilo lihih permutacij.
	\end{proof}

	\subsection{Izomorfizmi grup}
	\begin{defn}[Homomorfizem]
		Naj bosta $(G, \cdot)$ in $(H, *)$ grupe. Preslikava
		$\alpha G \rightarrow H$
		je homomorfizem, "ce
		\[ \forall a, b \in G: \alpha (a \cdot b) = \alpha(a) * \alpha (b) \]
	\end{defn}
	\begin{defn}[Avtomorfizem]
		Homomorfizem $G \rightarrow G$.
	\end{defn}
	\begin{defn}[Izomorfizem]
		Bijektivni homomorfizem.
	\end{defn}
	\begin{defn}[Izomorfni grupi]
		Grupi, med katerima obstaja izomorfizem.
	\end{defn}

	\begin{theorem}[Cayleyev]
		Vsaka grupa je izomorfna neki permutacijski grupi.
	\end{theorem}
	\begin{proof}
		Naj bo $G$ poljubna grupa in $g \in G$. Definirajmo $T_g: G \rightarrow G$:
		\[ T_g(x) = gx \]
		$T_g$ je permutacija mno"zice G. \\
		$H = \lbrace T_g: g \in G \rbrace$ je grupa za komponiranje. \\
		$H \cong G$
	\end{proof}

	\begin{claim}
		"Ce je $\alpha: G \rightarrow H$ izomorfizem grup, potem (med drugim) veljajo naslednje lastnosti:
		\begin{itemize}
			\item $\alpha$ preslika enoto $G$ v enoto $H$.
			\item "ce je $a \in G, a \in \Z \implies \alpha(a^n) = (\alpha(a))^n$
			\item "ce $a$ in $b$ komutirata v $G$, potem $\alpha(a)$ in $\alpha(b)$ komutirata v $H$.
			\item $G$ je abelova $\iff$ $H$ je abelova.
			\item $G$ je cikli"cna $\iff$ $H$ je cikli"cna.
			\item "ce je $K \leq G$, potem je $\alpha(K) = \lbrace \alpha(k): k \in K \rbrace \leq H$
		\end{itemize}
	\end{claim}

	\section{Odseki in pogrupe edinke}
	Naj bo $G$ grupa in $H \subseteq G$. Za $a \in G$ definirajmo:	
	\begin{defn}[Levi odsek $aH$]
		\[ aH = \lbrace ak: k \in H \rbrace \]
	\end{defn}
	\begin{defn}[Desni odsek $Ha$]
		\[ Ha = \lbrace ka: k \in H \rbrace \]
	\end{defn}

	\begin{ex}
		$G = S_3$. $H = \lbrace (1), (2) \rbrace $
		\begin{itemize}
			\item $(1)H = H$
			\item $(12)H = \lbrace (12)(1), (12)(12) \rbrace = \lbrace (12),(1)(2)(3)\rbrace = H $
			\item $(13)H = \lbrace (13)(1), (13)(12) \rbrace = \lbrace (13),(123) \rbrace $
			\item $(23)H = \lbrace (23)(1), (23)(12)\rbrace = \lbrace (23),(123) \rbrace $
			\item $(123)H = \lbrace (123)(1), (123)(12) \rbrace = \lbrace (123),(13) \rbrace $
			\item $(132)H = \lbrace (132)(1), (132)(12) \rbrace = \lbrace (132),(23) \rbrace $
		\end{itemize}
	\end{ex}
	\begin{ex}
		$G = (\Z_{10}, +)$. $H = (\lbrace 0, 2, 4, 6, 9 \rbrace, +)$
		\begin{itemize}
			\item $0+H = 2 + H = 4 + H = 6 + H = 8 + H$
			\item $1+H = 3 + H = 5 + H = 7 + H = 9 + H$
		\end{itemize}
	\end{ex}
	Ugotovitve: opazimo, da odseki niso nujno podgrupe $H$. Lahko se zgodi, da je $aH = bH$ za $a \neq b$ ($H(13) = (13)H$). $aH \neq Ha$ je povsem mo"zno.
	
	\begin{claim}[Najpomembnej"se lastnosti odsekov]\label{lastnosti_odsekov}
		Naj bo $H$ poljubna podgrupa grupe $G$, $a,b \in G$. Tedaj veljajo naslednje lastnosti:
		\begin{enumerate}
			\item $a \in aH \land a \in Ha$
			\item $aH = H \iff a \in H \iff Ha = H$
			\item bodisi $aH = Ha$ bodisi $aH \cap Ha = \emptyset$
			\item $aH = bH \iff a^{-1}b \in H \iff Ha = Hb$
			\item $|aH| = |bH| \land |Ha| = |Hb|$
			\item $aH = Ha \iff H = aHa^{-1}$
			\item $aH \leq G \iff a \in H \iff Ha \leq G$
		\end{enumerate}
	\end{claim}
	\begin{proof}
		Dokazali bomo prve tri trditve, ostale si boste pa sami.\\
		\begin{enumerate}
			\item $a \in aH$: $e \in H \implies a\cdot e \in aH$
			\item $aH = H \iff a \in H$:
			
				($\Rightarrow$) Naj velja $aH = H$. Ker je $a \in aH$ (po 1.) in ker je $aH = H$, je $a \in H$.
				
				($\Leftarrow$) Naj bo $a \in H$. Doka"zimo $aH = H$. \\
				Najprej $aH \subseteq H$: Naj bo $x \in aH$. Torej je $x = ak$ za nek $k \in H$.
				\[ a \in H, k \in H \implies ak \in H \]
				
				Sedaj "se $H \subseteq aH$: naj bo $k \in H$. Ker je $a \in H$, je \[ a^{-1} \in H \implies a^{-1}k \in H \]
				\[a(a^{-1}k) = k \in aH \]
			\item "Ce sta odseka disjunktna, ni kaj dokazovati. Recimo, da obstaja $x \in aH \cup bH$. $x \in aH \implies x = ak$ za nek $k \in H$. $x \in bH \implies x = bk'$ za nek $k' \in H$. Torej $ak = bk'$.
			\[ a = bk'k^{-1} \]
			\[ aH = (bk'k^{-1})H = (bk')(k^{-1}H) \]
			To"cka 2 pravi, da $k^{-1}H = H$ (ker je $k^{-1} \in H$).
			\[ aH = (bk')H = b(k'H) = bH \]
		\end{enumerate}
	\end{proof}

	"Ce zdru"zimo lastnosti 1, 2 in 5, ugotovimo, da levi odseki po podgrupi $H$ razdelijo grupo $G$ v (paroma disjunktne) bloke iste mo"ci.
	
	\begin{ex}
		$G = (\R^2, +)$. $H = $ premica skozi izhodi"s"ce.
		
		\[ (a,b) \in \R^2: (a,b)H = (a,b)+H = \lbrace (a+x, b+y): (x,y) \in H \rbrace \]
		Desni odseki po podgrupi $H$ (premica $p$) nam razdelijo ravnino v premice, ki so vzporedne s $p$.
	\end{ex}

	% 11. 05. 2021
	\begin{theorem}[Lagrange]
		Mo"c podgrupe deli mo"c grupe. "Stevilo razli"cnih levih (in desnih) odsekov po $H$ je $\frac{|G|}{|H|}$.
	\end{theorem}
	\begin{proof}
		Naj bodo $a_1H, \ldots, a_kH$ paroma razli"cni levi odseki podgrupe $H$. Tedaj velja:
		\[ |G| = |a_1H \cup \ldots \cup a_kH| \]
		To nam zagotavlja prva lastnost trditve \ref{lastnosti_odsekov} ($a \in aH$).
		\[ = |a_1H| + \ldots + |a_kH| \]
		(po lastnosti 3)
		\[ = k \cdot |H| \]
		(po lastnosti 5)
		\[ \implies k = \frac{|G|}{|H|} \]
	\end{proof}

	\begin{conseq}\label{red_deli_moc}
		Red elementa kon"cne grupe deli mo"c grupe.
	\end{conseq}
	\begin{proof}
		Vzemimo poljuben element $a \in G$ reda $n$.
		\[ \langle a \rangle = \lbrace e, a, \ldots, a^{n-1} \rbrace \leq G \overset{\text{lagrange}}{\implies} n = |\langle a \rangle| \operatorname{deli} |G| \]
	\end{proof}

	\begin{conseq}
		Grupa pra"stevilske mo"ci je cikli"cna.
	\end{conseq}
	\begin{proof}
		\[ |\langle a \rangle| \operatorname{deli} p \qquad |\langle a \rangle| \geq 2 \]
		Od tod sledi, da $|\langle a \rangle| = p$, torej $\langle a \rangle = G$.
	\end{proof}

	\begin{conseq}\label{a_na_moc}
		"Ce je $a$ element kon"cne grupe $G$, velja $a^{|G|} = e$.
	\end{conseq}
	\begin{proof}
		Po posledici \ref{red_deli_moc} $n$ deli $|G|$, torej $|G| = k \cdot n$.
		\[ a^{|G|} = a^{k \cdot n} = (a^n)^p = e \]
	\end{proof}

	\begin{conseq}[Mali Fermatov izrek]
		"Ce je $p$ pra"stevilo in $a \in \Z$, potem je
		\[ a^p \operatorname{mod} p = a \operatorname{mod} p \]
	\end{conseq}
	\begin{proof}
		$a = k \cdot p + r$, kjer $0 \leq r < p$.
		Naj bo $r = 0$: $a \operatorname{mod} p = 0$, $a^p \operatorname{mod} p = 0$.
		Naj bo $1 \leq r < p$: poglejmo grupo
		\[ G := (\Z_p - \lbrace 0 \rbrace, \cdot) \qquad |G| = p-1 \]
		Po posledici \ref{a_na_moc} velja $r^{p-1} = 1$, torej $r^p = r$.
	\end{proof}

	\subsection{Podgrupe edinke in faktorske grupe}
	\begin{defn}[Podgrupa edinka]
		Podgrupa $H$ je edinka, "ce velja
		\[ \forall a \in G: (aH = Ha) \]
		Ozna"cimo $H \triangleleft G$.
	\end{defn}

	Po to"cki 6 iz lastnosti odsekov (\ref{lastnosti_odsekov}) je torej
	\[ H \triangleleft G \iff H = aHa^{-1} \quad \forall a \in G \]
	
	\begin{claim}
		$aHa^{-1} \leq G$
	\end{claim}
	\begin{proof}
		\[ x,y \in aHa^{-1} \implies x^{-1}y \in aHa^{-1} \]
		\[ x = aka^{-1} \quad \text{za nek }k \in H \]
		\[ y = ak'a^{-1} \quad \text{za nek }k' \in H \]
		
		\[x^{-1}y = (aka^{-1})^{-1}(ak'a^{-1}) = (ak^{-1}a^{-1})(ak'a^{-1}) = a(k^{-1}k')a^{-1} \implies x^{-1}y \in aHa^{-1} \]
	\end{proof}

	\begin{ex}
		\[a = e \quad eHe^{-1} = \lbrace eke^{-1}: k\in H \rbrace = \lbrace k: k \in H \rbrace = H \]
	\end{ex}
	
	\begin{defn}[Konjugirana grupa]
		$aHa^{-1}$ je konjugirana grupa v $G$
	\end{defn}
	\begin{claim}
		$H \triangleleft G$, "ce je to edina mo"zna konjugirana grupa v $G$.
	\end{claim}

	\begin{defn}[Enostavna grupa]
		Je grupa, katere edini edinki sta $G$ in $\lbrace e \rbrace$.
	\end{defn}

	Osrednji razlog za pomembnost edink je to, da lahko iz odsekov edink tvorimo grupo.
	
	Naj bo $G$ grupa in $H \leq G$. Definirajmo mno"zico odsekov
	\[ G / H := \lbrace aH: a \in G \rbrace \]
	in vpeljimo operacijo
	\[ (aH)*(bH) := (ab)H \]
	\begin{theorem}
		"Ce je $H \triangleleft G$, potem je $(G / H, *)$ grupa.
	\end{theorem}
	\begin{proof}
		Vse lastnosti grupe zelo lahko sledijo iz definicije odseka in operacije med njimi.
		\begin{itemize}
			\item enota: $eH$
			\item inverz: $a^{-1}H$
			\item $\ldots$
		\end{itemize}
		Bistvo je, da poka"zemo, da je $*$ dobro definirana, t.j. da je rezultat neodvisen od izbire elementa iz odseka.
		
		Naj bosta $a$ in $a'$ iz istega odseka ($aH = a'H$) ter $b$ in $b'$ iz istega odseka ($bH = b'H$). Pokazati moramo, da je $(aH)*(bH)  = (a'H)*(b'H)$.
		
		\[ a' \in aH \implies a' = ak' \quad k' \in H \]
		\[ b' \in bH \implies b' = bk'' \quad k'' \in H \]
		\[ (a'H)*(b'H) \overset{\text{def.}}{=} (a'b')H = ak'bk''H = ak'b(k''H) \]
		\[ ak'(bH) \overset{\text{edinka}}{=} ak'(Hb) = a(k'H)b \overset{k' \in H}{=} aHb \]
		\[ a(Hb) \overset{\text{edinka}}{=} a(bH) \overset{\text{def.}}{=} (aH)*(bH) \]
	\end{proof}

	\begin{defn}[Faktorska grupa grupe $G$ po edinki $H$]
		Grupa $(G / H, *)$ po zgoraj definiranih operacijah $*$ in $/$.
	\end{defn}

	\begin{theorem}
		"Ce je $G$ grupa in $G/Z(G)$ cikli"cna grupa, potem je $G$ abelova.
	\end{theorem}
	\begin{proof}
		QED.
		% kle je reku da ne bomo dokazal, ampak je useen narisu kvadratek, tkoda I guess da to steje kot dokaz
	\end{proof}

	\section{Kolobarji in polja}
	\begin{rem}
		Hi, author here. V naslednjem razdelku spu"s"cam nekatere dokaze in primere, ker so bodisi zelo trivialni, ali pa smo jih "ze videli pri Linearni algebri. Spu"s"ceni dokazi so ozna"ceni z ``Redacted''.
		Author out.
	\end{rem}

	\begin{defn}[Kolobar]
		Mno"zica z 2 operacijama $(R, +, \cdot)$ \\
		kjer je $(R, +)$ abelova grupa in $(R, \cdot)$ polgrupa.
		
		Velja distributivnost mno"zenja prek se"stevanja:
		\[ a(b+c) = ab + ac \quad \land \quad (a+b)c = ac + bc \]
		% kokoid
	\end{defn}
	
	\begin{defn}[Komutativen kolobar]
		Kolobar, v katerem je mno"zenje komutativno.
	\end{defn}
	\begin{ex}
		$2\Z$ soda cela "stevila.
	\end{ex}
	\begin{defn}[Kolobar z enoto]
		Kolobar, v katerem obstaja enota za mno"zenje.
	\end{defn}
	\begin{ex}
		$M_2(\Z)$ 2x2 matrike z elementi iz $\Z$.
	\end{ex}
	\begin{defn}[Kokoid]
		% KOKOID
		Komutativen kolobar z identiteto (enoto).
	\end{defn}

	\begin{defn}[Direktna vsota]
		\[ (R, +_{R}, \cdot_{R}) \oplus (S, +_{S}, \cdot_{S}) := (R \times S, +_{R \times S}, \cdot_{R \times S}) \]
		\[ (r,s) +_{R \times S} (r',s') := (r+_{R}r', s+_{S}s') \]
		\[ (r,s) \cdot_{R \times S} (r',s') := (r \cdot_{R} r', s \cdot_{S} s') \]
	\end{defn}
	\begin{theorem}
		"Ce sta $R$ in $S$ kolobarja, je $R \oplus S$ kolobar. "Ce imata enoto, jo ima tudi produkt. "Ce sta komutativna, je tak tudi produkt.
	\end{theorem}
	\begin{proof}
		Z enostavnim izra"cunom.
	\end{proof}
	
	\begin{rem}
		Konstrukcijo lahko raz"sirimo na direktne vsote kon"cnega "stevila kolobarjev: $R_1 \oplus R_2 \oplus \ldots R_n$.
		To je v bistvu posplo"sitev $\R^n$.
	\end{rem}
	
	\subsection{Lastnosti kolobarjev}	
	\begin{itemize}
		\item Nevtralni element za $+$, torej $0$, je enoli"cen.
		\item "Ce je $R$ kolobar z enoto $1$, je tudi ta enoli"cna.
	\end{itemize}
	\begin{theorem}
		Naj bo $R$ kolobar in $a,b \in R$. Potem velja:
		\begin{enumerate}
			\item $0 \cdot a = a \cdot 0 = 0$
			\item $(-a) \cdot b = a \cdot (-b) = -(a \cdot b)$
			\item $(-a) \cdot (-b) = a \cdot b$
		\end{enumerate}
	\end{theorem}
	\begin{proof}
		Redacted.
	\end{proof}

	\begin{conseq}
		"Ce ima kolobar enoto $1$, velja $(-1) \cdot a = -(1 \cdot a) = -a$
	\end{conseq}

	\subsection{Podkolobarji}
	\begin{defn}
		Naj bo $R$ kolobar in $S \subseteq R$. "Ce je $S$ kolobar za isti operaciji kot jih ima $R$, je $S$ podkolobar kolobarja $R$.
	\end{defn}
	\begin{ex}
		$\Z \subseteq \Q$
	\end{ex}
	\begin{ex}
		$\Q \subseteq \R$
	\end{ex}
	\begin{ex}
		$n \geq 2 \quad n\Z \subseteq \Z$
	\end{ex}

	\begin{theorem}
		$S$ je podkolobar $R$ natanko tedaj, ko velja vse izmed:
		\begin{itemize}
			\item $S \subseteq R$
			\item $0 \in S$
			\item $\forall a,b \in S: a-b \in S$
			\item $\forall a,b \in S: a-b \in S$
		\end{itemize}
	\end{theorem}
	\begin{proof}
		Redacted.
	\end{proof}

	\begin{defn}[Center kolobarja]
		Je mno"zica tistih elementov, ki komutirajo z vsemi elementi.
		\[ \lbrace x \in R: ax = xa \quad \forall x \in R \rbrace \]
	\end{defn}
	\begin{claim}
		Center kolobarja je njegov podkolobar.
	\end{claim}
	\begin{proof}
		Redacted.
	\end{proof}

	%\subsection{Delitelji ni"ca in celi kolobarji}
	%\subsection{Polja}
	%\subsection{Podpolja}
	%\subsection{Karakteristika kolobarja}
	%\subsection{Ideali}
\end{document}
